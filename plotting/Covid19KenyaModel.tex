\documentclass[a4paper, 11pt]{article}


%%%%%%%%%%%%%%%%%%%%%%%%%%%%%%%%%%%%%%%%%%%%%%%%%%%%%%%%%%%%%%%%%%%%%%%%%%%%%%%

%Standard Maths Packages

\usepackage{amsmath, amssymb, parskip}

\usepackage{array,graphicx}
\usepackage{booktabs}
\usepackage{pifont}
\usepackage[table]{xcolor}
\newcolumntype{P}[1]{>{\centering\arraybackslash}p{#1}}

\newcommand*\rot{\rotatebox{90}}
\newcommand*\rottwo{\rotatebox[origin=rc]{90}}
\newcommand*\OK{\ding{51}}
\usepackage{pdflscape}
\usepackage[normalem]{ulem}	

\renewcommand*{\familydefault}{\sfdefault} %Changes font to serif

%%%%%%%%%%%%%%%%%%%%%%%%%%%%%%%%%%%%%%%%%%%%%%%%%%%%%%%%%%%%%%%%%%%%%%%%%%%%%%%

%Redefining numbered sections to alphabetic ones

\def\thepart{\Alph{part}} 
\usepackage{setspace}
\renewcommand{\thesection}{S\arabic{section}}
\renewcommand{\thefigure}{S\arabic{figure}}
\renewcommand{\thetable}{S\arabic{table}}

%%%%%%%%%%%%%%%%%%%%%%%%%%%%%%%%%%%%%%%%%%%%%%%%%%%%%%%%%%%%%%%%%%%%%%%%%%%%%%%

%Graphics packages
\usepackage{graphicx,caption,subcaption}


%%%%%%%%%%%%%%%%%%%%%%%%%%%%%%%%%%%%%%%%%%%%%%%%%%%%%%%%%%%%%%%%%%%%%%%%%%%%%%%

%%%%%%%%%%%%%%%%%%%%%%%%%%%%%%%%%%%%%%%%%%%%%%%%%%%%%%%%%%%%%%%%%%%%%%%%%%%%%%%

%Tikzpicture to draw nice vector-scaled diagrams in LaTex

\usepackage{tikz}
\usetikzlibrary{shapes,arrows}
\usetikzlibrary{decorations.markings}

%%%%%%%%%%%%%%%%%%%%%%%%%%%%%%%%%%%%%%%%%%%%%%%%%%%%%%%%%%%%%%%%%%%%%%%%%%%%%%%

%Margins

\usepackage[paperwidth=16cm, paperheight=8cm,margin=1cm]{geometry}

 \usepackage{nopageno}
 \pagestyle{plain}

%%%%%%%%%%%%%%%%%%%%%%%

\begin{document}

\tikzstyle{block} = [rectangle, draw, fill=white, 
    text width=3em, text centered, rounded corners, minimum height=3em]


\tikzstyle{block2} = [rectangle, draw, fill=yellow!80, 
    text width=3em, text centered,node distance=4cm, rounded corners, minimum height=3em]

\tikzstyle{block3} = [rectangle, draw, fill=red!30, 
    text width=3em, text centered, rounded corners, minimum height=3em]

 
\tikzstyle{line} = [draw, -triangle 45]
   \tikzstyle{-->}=[draw,dashed,decoration={markings,mark=at position 0.99 with {\arrow[scale=2]{angle 90}}},
    postaction={decorate},
    shorten >=0.4pt]


\begin{figure}
\begin{centering}    
\begin{tikzpicture}[node distance = 2cm, auto]

   % Place nodes
    \node [block] (S) at (0,0) {$S_{i,a}$};
    \node [block] (E) at (3,0) {$E_{i,a}$};
    \node [block3] (D) at (6,2) {$D_{i,a}$};
    \node [block2] (A) at (6,-2) {$A_{i.a}$};
    \node [block] (Q) at (9,2) {$Q_{i,a}$};
    \node [block] (R) at (12,0) {$R_{i,a}$};
    
    \path [line] (S) to node {FOI} (E);
    
    \node [anchor=mid] at (1,3) {$\mbox{FOI}= \sum_j T_{ij,a}\lambda_{j,a}$};
     \path [line] (E) to node {$\sigma \delta_a$} (D);
      \path [line] (E) to node [below, near start] {} (A);
      \node [anchor=mid] at (4,-1.5) {$\sigma (1-\delta_a)$};
       \path [line] (D) to node {$\tau(t)$} (Q);
        \path [line] (Q) to node {$\varphi$} (R);
         \path [line] (A) to node {$\gamma $} (R);
          \path [line] (D) to  [out=315,in=180,looseness=0.7] node {$\gamma$} (R);
          
          \path [-->] (A) to  [out=210,in=270,looseness=0.5]  (1.5,-0.5);
          \node [anchor=mid] at (3,-2.3) {$\times \varepsilon_A$};
          
          \path [-->] (D) to  [out=150,in=90,looseness=0.5]  (1.5,0.7);
           \node [anchor=mid] at (3,2.3) {$\times \varepsilon_D$};
          
\end{tikzpicture}
\end{centering}
\end{figure}


\end{document}
